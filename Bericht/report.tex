\documentclass[a4paper,12pt]{article}

% Packages
\usepackage[utf8]{inputenc}
\usepackage{amsmath}
\usepackage{graphicx}
\usepackage{hyperref}
\usepackage{geometry}
\geometry{a4paper, margin=1in}

% Title and Author
\title{Data Mining Report}
\author{Ilhan Arlsan, Zenel Boshnjaku, Eliane Hess, Simeon Hacker}
\date{\today}

\begin{document}

\maketitle
\tableofcontents
\newpage

\section{Einführung}
\label{sec:Einführung}
% Brief overview of data mining, its importance, and the purpose of the report.

% Create subsection to detail the available data and the research questions.
\subsection{Verfügbare Daten und Forschungsfragen}
\label{subsec:Verfügbare Daten und Forschungsfragen}

Die verfügbaren Daten sind im wesentlichen drei .csv-Dateien, die die folgenden Informationen enthalten:
Diese Datei enthält aufgezeichnete Daten in Anlage 1. Es umfasst 1~Mio Dateneinträge mit insgesamt 6 Parametern.



\section{Forschungsfrage I - Data Mining}
\label{sec:FI}
% Discuss the theoretical background of data mining and related research.

\section{Forschungsfrage II - Process Mining}
\label{sec:FII}
% Explain the methods and techniques used in the data mining process.

\appendix
\section{Anhang}
\label{sec:anhang}
% Include any additional material, such as code or detailed calculations.

\bibliographystyle{plain}
\bibliography{references}

\end{document}